%%%%%%%%%%%%%%%%%%%%%%%%%%%%%%%%%%%%%%%%%
% LaTeX Template
%
% This template originates from:
% http://www.LaTeXTemplates.com
%
% License:
% CC BY-NC-SA 3.0 (http://creativecommons.org/licenses/by-nc-sa/3.0/)
% 
%%%%%%%%%%%%%%%%%%%%%%%%%%%%%%%%%%%%%%%%%

%----------------------------------------------------------------------------------------
%	PACKAGES AND OTHER DOCUMENT CONFIGURATIONS
%----------------------------------------------------------------------------------------

\documentclass{article}

\input{structure.tex} % Include the file specifying the document structure and custom commands

\usepackage{graphicx}
%Path in Windows format:
\graphicspath{ {.Images/} }

%----------------------------------------------------------------------------------------
%	ASSIGNMENT INFORMATION
%----------------------------------------------------------------------------------------

\title{Gambler's Ruin - Theory} % Title of the assignment

\author{Tansel Arif\\ \texttt{tanselarif@live.co.uk}} % Author name and email address

\date{\today} % University, school and/or department name(s) and a date

%----------------------------------------------------------------------------------------

\begin{document}

\maketitle % Print the title

%----------------------------------------------------------------------------------------
%	INTRODUCTION
%----------------------------------------------------------------------------------------

\section*{Introduction} % Unnumbered section

The Gambler's Ruin problem frames a gambler who begins gambling with an initial fortune - in dollars say. At each successive gamble, the gambler either loses \$1 or gains \$1. The problem is to find the probability that the gambler goes bankrupt - loses the entirety of the fortune. This problem is a kind of random walk. Figures \ref{fig:sim1} and \ref{fig:sim2} below show simulation trajectories for this setup.

\begin{figure}[r]
    \centering
    \includegraphics[width=\textwidth]{Simulation1}
    \caption{This is a plot of 10 simulations with $k = 12.5$, $p = 0.55$, $N = 25$. The theory results in a probability of 0.0753 of bankruptcy. The horizontal green line represents \$N and the hosrizontal red line represents \$0.}
    \label{fig:sim1}
\end{figure}

\begin{figure}[r]
    \centering
    \includegraphics[width=\textwidth]{Simulation2}
    \caption{This is a plot of 1000 simulations with $k = 12.5$, $p = 0.55$, $N = 25$. The theory results in a probability of 0.0753 of bankruptcy. The horizontal green line represents \$N and the hosrizontal red line represents \$0.}
    \label{fig:sim2}
\end{figure}

\section{The Problem} % Numbered section

A Gambler begins with \$k and repeatedly plays a game after which they may win \$1 with probability $p$ or lose \$1 with probability $q=1-p$. The Gambler will stop playing if their fortune reaches \$0 or \$N. What is the probability that they go bankrupt?

\section{The Solution} % Numbered section

Let $u_k$ be the probability that the Gambler bankrupts if the initial fortune is \$k. Then we can condition this probability on the first gamble as follows (utilising the law of total probability with the partitioning of lose/win):

$$u_k = P(wins) \times u_{k+1} + P(loses) \times u_{k-1}$$

This is a second order homogeneous difference equation. We look for solutions of the form $u_n = A \times \lambda^n$.

$$p \times u_{n+1} - u_n + q u_{n-1} = 0$$
$$\implies p \times A \times \lambda^{n+1} - A \times \lambda^n + q \times A \times \lambda^{n-1} = 0$$
$$\implies \lambda^{2} - \frac{1}{p} \lambda + \frac{q}{p} = 0$$

where $p, q \neq 0$. This has solution:

$$\lambda_{1,2} = \left\{\frac{1-p}{p},1\right\}$$

provided that $p \neq \frac{1}{2}$, this gives 2 different solutions. We have:

$$u_n = A \left( \frac{1-p}{p} \right)^n + B (1)^n$$
$$ = A \left( \frac{1-p}{p} \right)^n + B$$

We have that the Gambler stops gambling if either their fortune reaches \$0 or \$N. So we have the following boundary conditions:

$$u_0 = 1, u_N = 0$$

Using these boundary conditions, we can solve for $A$ and $B$:

$$u_0 = A \left( \frac{1-p}{p} \right)^0 + B = 1$$
$$\implies A+B = 1$$
$$\implies B = 1 - A$$

and

$$u_N = A \left( \frac{1-p}{p} \right)^N + B = 0$$
$$\implies B = -A \left( \frac{1-p}{p} \right)^N$$
$$\implies 1 - A = -A \left( \frac{1-p}{p} \right)^N$$
$$\implies A = \frac{1}{1 - \left( \frac{1-p}{p} \right)^N}$$
$$\implies B = 1-A = \frac{-\left( \frac{1-p}{p} \right)^N}{1 - \left( \frac{1-p}{p} \right)^N}$$

Giving the final solution:

\begin{equation}
    u_n = \frac{\left( \frac{1-p}{p} \right)^n-\left( \frac{1-p}{p} \right)^N}{1 - \left( \frac{1-p}{p} \right)^N}
\end{equation}

For the case where $p = \frac{1}{2}$, we try the next most complex expression, let:

$$u_n = (An + B) \times \lambda^n$$

with $\lambda = 1$:

$$u_n = (An + B)$$

We can try this in the original equation with $p=q=1/2$:

$$\frac{1}{2} u_{n+1} - u_n + \frac{1}{2} u_{n-1} = \frac{An}{2} + \frac{A}{2} + \frac{B}{2} - An -B + \frac{An}{2} - \frac{A}{2} + \frac{B}{2} = 0$$

Using the boundary conditions:

$$u_0 = B = 1$$

$$u_N = AN + B = 0 \implies A = \frac{-1}{N}$$

Giving the final equation as:

\begin{equation}
    u_n = 1 - \frac{n}{N}
\end{equation}

%----------------------------------------------------------------------------------------

\end{document}
